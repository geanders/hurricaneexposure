\documentclass[]{elsarticle} %review=doublespace preprint=single 5p=2 column
%%% Begin My package additions %%%%%%%%%%%%%%%%%%%
\usepackage[hyphens]{url}
\usepackage{lineno} % add
\providecommand{\tightlist}{%
  \setlength{\itemsep}{0pt}\setlength{\parskip}{0pt}}

\bibliographystyle{elsarticle-harv}
\biboptions{sort&compress} % For natbib
\usepackage{graphicx}
\usepackage{booktabs} % book-quality tables
%% Redefines the elsarticle footer
%\makeatletter
%\def\ps@pprintTitle{%
% \let\@oddhead\@empty
% \let\@evenhead\@empty
% \def\@oddfoot{\it \hfill\today}%
% \let\@evenfoot\@oddfoot}
%\makeatother

% A modified page layout
\textwidth 6.75in
\oddsidemargin -0.15in
\evensidemargin -0.15in
\textheight 9in
\topmargin -0.5in
%%%%%%%%%%%%%%%% end my additions to header

\usepackage[T1]{fontenc}
\usepackage{lmodern}
\usepackage{amssymb,amsmath}
\usepackage{ifxetex,ifluatex}
\usepackage{fixltx2e} % provides \textsubscript
% use upquote if available, for straight quotes in verbatim environments
\IfFileExists{upquote.sty}{\usepackage{upquote}}{}
\ifnum 0\ifxetex 1\fi\ifluatex 1\fi=0 % if pdftex
  \usepackage[utf8]{inputenc}
\else % if luatex or xelatex
  \usepackage{fontspec}
  \ifxetex
    \usepackage{xltxtra,xunicode}
  \fi
  \defaultfontfeatures{Mapping=tex-text,Scale=MatchLowercase}
  \newcommand{\euro}{€}
\fi
% use microtype if available
\IfFileExists{microtype.sty}{\usepackage{microtype}}{}
\usepackage{color}
\usepackage{fancyvrb}
\newcommand{\VerbBar}{|}
\newcommand{\VERB}{\Verb[commandchars=\\\{\}]}
\DefineVerbatimEnvironment{Highlighting}{Verbatim}{commandchars=\\\{\}}
% Add ',fontsize=\small' for more characters per line
\usepackage{framed}
\definecolor{shadecolor}{RGB}{248,248,248}
\newenvironment{Shaded}{\begin{snugshade}}{\end{snugshade}}
\newcommand{\KeywordTok}[1]{\textcolor[rgb]{0.13,0.29,0.53}{\textbf{{#1}}}}
\newcommand{\DataTypeTok}[1]{\textcolor[rgb]{0.13,0.29,0.53}{{#1}}}
\newcommand{\DecValTok}[1]{\textcolor[rgb]{0.00,0.00,0.81}{{#1}}}
\newcommand{\BaseNTok}[1]{\textcolor[rgb]{0.00,0.00,0.81}{{#1}}}
\newcommand{\FloatTok}[1]{\textcolor[rgb]{0.00,0.00,0.81}{{#1}}}
\newcommand{\ConstantTok}[1]{\textcolor[rgb]{0.00,0.00,0.00}{{#1}}}
\newcommand{\CharTok}[1]{\textcolor[rgb]{0.31,0.60,0.02}{{#1}}}
\newcommand{\SpecialCharTok}[1]{\textcolor[rgb]{0.00,0.00,0.00}{{#1}}}
\newcommand{\StringTok}[1]{\textcolor[rgb]{0.31,0.60,0.02}{{#1}}}
\newcommand{\VerbatimStringTok}[1]{\textcolor[rgb]{0.31,0.60,0.02}{{#1}}}
\newcommand{\SpecialStringTok}[1]{\textcolor[rgb]{0.31,0.60,0.02}{{#1}}}
\newcommand{\ImportTok}[1]{{#1}}
\newcommand{\CommentTok}[1]{\textcolor[rgb]{0.56,0.35,0.01}{\textit{{#1}}}}
\newcommand{\DocumentationTok}[1]{\textcolor[rgb]{0.56,0.35,0.01}{\textbf{\textit{{#1}}}}}
\newcommand{\AnnotationTok}[1]{\textcolor[rgb]{0.56,0.35,0.01}{\textbf{\textit{{#1}}}}}
\newcommand{\CommentVarTok}[1]{\textcolor[rgb]{0.56,0.35,0.01}{\textbf{\textit{{#1}}}}}
\newcommand{\OtherTok}[1]{\textcolor[rgb]{0.56,0.35,0.01}{{#1}}}
\newcommand{\FunctionTok}[1]{\textcolor[rgb]{0.00,0.00,0.00}{{#1}}}
\newcommand{\VariableTok}[1]{\textcolor[rgb]{0.00,0.00,0.00}{{#1}}}
\newcommand{\ControlFlowTok}[1]{\textcolor[rgb]{0.13,0.29,0.53}{\textbf{{#1}}}}
\newcommand{\OperatorTok}[1]{\textcolor[rgb]{0.81,0.36,0.00}{\textbf{{#1}}}}
\newcommand{\BuiltInTok}[1]{{#1}}
\newcommand{\ExtensionTok}[1]{{#1}}
\newcommand{\PreprocessorTok}[1]{\textcolor[rgb]{0.56,0.35,0.01}{\textit{{#1}}}}
\newcommand{\AttributeTok}[1]{\textcolor[rgb]{0.77,0.63,0.00}{{#1}}}
\newcommand{\RegionMarkerTok}[1]{{#1}}
\newcommand{\InformationTok}[1]{\textcolor[rgb]{0.56,0.35,0.01}{\textbf{\textit{{#1}}}}}
\newcommand{\WarningTok}[1]{\textcolor[rgb]{0.56,0.35,0.01}{\textbf{\textit{{#1}}}}}
\newcommand{\AlertTok}[1]{\textcolor[rgb]{0.94,0.16,0.16}{{#1}}}
\newcommand{\ErrorTok}[1]{\textcolor[rgb]{0.64,0.00,0.00}{\textbf{{#1}}}}
\newcommand{\NormalTok}[1]{{#1}}
\usepackage{graphicx}
% We will generate all images so they have a width \maxwidth. This means
% that they will get their normal width if they fit onto the page, but
% are scaled down if they would overflow the margins.
\makeatletter
\def\maxwidth{\ifdim\Gin@nat@width>\linewidth\linewidth
\else\Gin@nat@width\fi}
\makeatother
\let\Oldincludegraphics\includegraphics
\renewcommand{\includegraphics}[1]{\Oldincludegraphics[width=\maxwidth]{#1}}
\ifxetex
  \usepackage[setpagesize=false, % page size defined by xetex
              unicode=false, % unicode breaks when used with xetex
              xetex]{hyperref}
\else
  \usepackage[unicode=true]{hyperref}
\fi
\hypersetup{breaklinks=true,
            bookmarks=true,
            pdfauthor={},
            pdftitle={Assessing exposure to Atlantic Basin tropical storms in United States counties},
            colorlinks=true,
            urlcolor=blue,
            linkcolor=magenta,
            pdfborder={0 0 0}}
\urlstyle{same}  % don't use monospace font for urls
\setlength{\parindent}{0pt}
\setlength{\parskip}{6pt plus 2pt minus 1pt}
\setlength{\emergencystretch}{3em}  % prevent overfull lines
\setcounter{secnumdepth}{0}
% Pandoc toggle for numbering sections (defaults to be off)
\setcounter{secnumdepth}{0}
% Pandoc header


\usepackage[nomarkers]{endfloat}

\begin{document}
\begin{frontmatter}

  \title{Assessing exposure to Atlantic Basin tropical storms in United States
counties}
    \author[Colorado State University]{Joshua Ferreri}
   \ead{alice@example.com} 
  
    \author[Colorado State University]{Meilin Yan}
   \ead{alice@example.com} 
  
    \author[NASA Marshall Space Flight Center]{Mohammad Z. Al-Hamdan}
   \ead{alice@example.com} 
  
    \author[NASA Marshall Space Flight Center]{William L. Crosson}
   \ead{alice@example.com} 
  
    \author[University of Michigan]{Seth Guikema}
   \ead{alice@example.com} 
  
    \author[Johns Hopkins Bloomberg School of Public Health]{Roger D. Peng}
   \ead{alice@example.com} 
  
    \author[Colorado State University]{G. Brooke Anderson\corref{c1}}
   \ead{brooke.anderson@colostate.edu} 
   \cortext[c1]{Corresponding Author}
      \address[Colorado State University]{Department of Environmental and Radiological Health Sciences, Lake
Street, Fort Collins, CO, Zip}
    \address[NASA Marshall Space Flight Center]{Universities Space Research Association, Street, Huntsville, AL, Zip}
    \address[University of Michigan]{Department, Street, City, State, Zip}
    \address[Johns Hopkins Bloomberg School of Public Health]{Department of Biostatistics, Street, Baltimore, MD, Zip}
  
  \begin{abstract}
  There are many important applications for having county-level estimates
  of exposure to tropical storms over many years. For example, \ldots{} .
  Current approaches include \ldots{} but have the following limitations
  \ldots{} .
  
  Here, we present open source software we have developed to explore
  county-level exposure to tropical storms in United States counties
  between 1988 and 2011. Further, we explore the differences in exposure
  classification when using different metrics (e.g., wind speed, rainfall,
  distance).
  \end{abstract}
  
 \end{frontmatter}

\begin{Shaded}
\begin{Highlighting}[]
\KeywordTok{library}\NormalTok{(hurricaneexposure)}
\end{Highlighting}
\end{Shaded}

\section{Introduction}\label{introduction}

\paragraph{What it means to assess county-level tropical storm
exposure}\label{what-it-means-to-assess-county-level-tropical-storm-exposure}

\paragraph{Why it's important to measure TS exposure for
counties}\label{why-its-important-to-measure-ts-exposure-for-counties}

\paragraph{Examples of using TS exposure estimates for multiple
storms}\label{examples-of-using-ts-exposure-estimates-for-multiple-storms}

\paragraph{Examples of other datasets at the county
level}\label{examples-of-other-datasets-at-the-county-level}

If exposure to tropical storms over multiple storms and years can be
assessed, these exposure datasets can be joined with other time series
to explore the impacts of tropical storms. For example, daily counts of
human health outcomes in environmental epidemiology studies are often
available aggregated at the county level, and such data has often been
paired with time series of environmental exposures (e.g., air pollution,
temperature) to determine associated risks.

\section{Data and Methods}\label{data-and-methods}

\paragraph{Distance-based exposure}\label{distance-based-exposure}

We collected ``best tracks'' data on hurricane tracks for Atlantic basin
storms between 1988 and 2014 from the extended best tracks database.
\ldots{} This data gives time stamps for each observation in UTC.

These data typically give measurements of storm center location at
6-hour intervals. We interpolated these location values to every 15
minutes during the period when the storm was active, using a linear
interpolation between each measured point.

We calculated the distance between each county's center and each of the
15-minute-interval estimates of a storm's location. To do this, we used
the Great Circle method to calculate the distance between pairs of
latitude and longitude coordinates, using the R package \texttt{sp}. For
each county, we used the population mean center, based on the U.S.
Census's 2010 Decennial Census, as the center location of the county.
From these distance calculations, we identified the date-time and
distance between the storm track and the county center at the 15-minute
interval when the storm was closest to the county center. We used the
\texttt{countytimezones} R package to convert each date-time of the
closest storm approach for a county to the county's local time zone and
identified the local date when the storm was closest to the county. This
``closest date'' was used to pair the distance estimates with rainfall
estimates.

\paragraph{Rain-based exposure}\label{rain-based-exposure}

\paragraph{Wind-based exposure}\label{wind-based-exposure}

\section{Results}\label{results}

\includegraphics{DraftExposurePaper_files/figure-latex/unnamed-chunk-2-1.pdf}

\section{Discussion}\label{discussion}

\paragraph{Distance-based exposure}\label{distance-based-exposure-1}

There are some sources of uncertainty for storm locations from the best
track hurricane data. These include \ldots{}

However, these best tracks should be fairly reliable for more recent
years of storms, as we use here. Many of the uncertainties related to
storm positions in best track data are more of a concern for the years
before \ldots{} (e.g., pre-19{[}xx{]}).

While there are often several different ``best tracks'' datasets, from
different sources {[}?-- weather services?{]} for other ocean basins,
the ``best tracks'' data from \ldots{}, which is the basis of the tracks
we use here, are the undisputed primary source of tropical storm track
data for the Atlantic basin.

There are a number of limitations to using distance to assess exposure
to tropical storms. Tropical storms vary in size and intensity, and a
measure of distance from the storm track will not incorporate these
differences and so could mis-classify exposure both in terms of
generating false positives (counties close to the storm track of very
mild or small-radius storms) and false negatives (e.g., during very
large or intense storms).

\paragraph{Rain-based exposure}\label{rain-based-exposure-1}

It can be very difficult to reliabily measure rain during extreme rain
events, including tropical storms. For example, a heavy rain can wash
away {[}?{]} rain monitors {[}?{]}. It can also be very hard to measure
rain during heavy wind, as the rain does not fall straight into the
monitor {[}?{]}.

Some of the other possible sources for estimating rain during tropical
storms include \ldots{}

The estimated rainfall amounts from our data are likely underestimates.
This data source, however, should be internally consistent and so useful
for comparing across different storms when all exposure estimates are
based on this rain data.

Rainfall estimates are likely underestimates for a few reasons. First,
they are based on averaging hourly measurements to a daily mean
estimate. This averaging would smooth over shorter periods of very
extreme rainfall. Further, this data is averaged over multiple grid
points within each county and so would not fully reflect very extreme
local precipitation (although this might be less of a concern for
classifying exposure to a large-scale storm system, like a tropical
storm, compared to more fine-scale storms). Finally, this NLDAS data
provides a re-analysis that incorporates measured rainfall, using
models, etc., to incorporate that observed data into a spatially and
temporally continuous dataset of rainfall. However, during extreme
storms, the problems with measuring rainfall using {[}rain monitors{]}
would propogate into the NLDAS data, so although NLDAS would prevent
missing values during the storm if monitors are not able to provide
data, if monitors are out, rainfall estimates from NLDAS will be based
more on models than on observations.

Exposure classification based on rainfall has some advantages. For
example, it allows the identification of exposed counties that are
inland, rather than coastal, but that were affected by heavy rainfall
during the storm. Often, storm-related deaths are associated with inland
flooding. Storms can produce a lot of rain especially in certain
topographies, like near mountains, so counties that are well inland can
sometimes experience more extreme rain that other counties at similar
distances from the storm's track.

In comparison to the storm track, the areas of extreme rain and extreme
wind can vary. For example, storms undergoing an extratropical
transition can bring heavy rains to the left of the storm's center
track, while extreme winds are more common to the right of the track
(Halverson 2015).

Some storms can be of lower intensity (i.e., on the Saffir-Simpson
scale) yet bring dangerous rainfall, including well inland of the
storm's landfall. For example, storms including Floyd in 1999, Gaston in
2004, Irene in 2011, and Lee in 2011, have had severe inland impacts,
often through extreme rainfall, as post-tropical storms
{[}Halverson2015{]}. This rainfall can be particularly severe in the
Appalachians {[}Halverson2015{]}.

For example,

\includegraphics{DraftExposurePaper_files/figure-latex/unnamed-chunk-3-1.pdf}

\paragraph{Wind-based exposure}\label{wind-based-exposure-1}

Wind suffers from similar challenges for measuring during tropical
storms. In particular, the strong winds of tropical storms can break or
blow away the anemometers used to measure wind speed.

Here, we used wind speed models, rather than observed wind speed, to
estimate exposure to tropical storms based on winds.

A variety of other wind speed models exist besides the one used here. In
particular, there are options for wind speed models as far as \ldots{}

\paragraph{Example uses of exposure
datasets}\label{example-uses-of-exposure-datasets}

\section*{References}\label{references}
\addcontentsline{toc}{section}{References}

\hypertarget{refs}{}
\hypertarget{ref-Halverson2015}{}
Halverson, Jeffrey B. 2015. ``Second Wind: The Dealdy and Destructive
Inland Phase of East Coast Hurricanes.'' \emph{Weatherwise} 68 (2):
20--27.
doi:\href{https://doi.org/10.1080/00431672.2015.997562}{10.1080/00431672.2015.997562}.

\end{document}


